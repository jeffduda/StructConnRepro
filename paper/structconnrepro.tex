%%%%%%%%%%%%%%%%%%%%%%%%%%%%%%%%%%%%%%%%%%%%%%%%%%%%%%%%%%%%%%%%%%%%%%%%%%%%%%%%%%%%%%%%%%%%%%%%%%%%%%%%%%%%%%%%%%%%%%%%%%%%%%%%%%%%%%%%
% This is just a template to use when submitting manuscripts to Frontiers, it is not mandatory to use frontiers.cls nor frontiers.tex  %
%%%%%%%%%%%%%%%%%%%%%%%%%%%%%%%%%%%%%%%%%%%%%%%%%%%%%%%%%%%%%%%%%%%%%%%%%%%%%%%%%%%%%%%%%%%%%%%%%%%%%%%%%%%%%%%%%%%%%%%%%%%%%%%%%%%%%%%%

\documentclass{frontiersSCNS} % for Science articles
%\documentclass{frontiersMED} % for Medicine articles

\usepackage{url}
\usepackage{tabularx}
\usepackage{lineno}
\usepackage{multirow}
\usepackage[english]{babel}
\usepackage{xcolor,colortbl}

\linenumbers

\newcommand{\R}{R}

\copyrightyear{}
\pubyear{}
%\onecolumn
%%% write here for which journal %%%
\def\journal{Neurosciences}
\def\DOI{}
\def\articleType{}
\def\citing{\color{darkgray}\cite}
\def\keyFont{\fontsize{6}{11}\helveticabold }
\def\firstAuthorLast{Duda {et~al}} %use et al only if is more than 1 author
\def\Authors{Jeffrey T. Duda\,$^{1,*}$, Philip A. Cook\,$^{1}$ and James C. Gee\,$^1$}
% Affiliations should be keyed to the author's name with superscript numbers and be listed as follows: Laboratory, Institute, Department, Organization, City, State abbreviation (USA, Canada, Australia), and Country (without detailed address information such as city zip codes or street names).
% If one of the authors has a change of address, list the new address below the correspondence details using a superscript symbol and use the same symbol to indicate the author in the author list.
\def\Address{$^{1}$Penn Image Computing and Science Laboratory, University of Pennsylvania, Department of Radiology, Philadelphia, PA, USA}
% The Corresponding Author should be marked with an asterisk
% Provide the exact contact address (this time including street name and city zip code) and email of the corresponding author
\def\corrAuthor{Jeffrey T. Duda}
\def\corrAddress{Penn Image Computing and Science Laboratory, University of Pennsylvania, Department of Radiology, 3600 Market Street, Suite 370, Philadelphia, PA, USA}
\def\corrEmail{jtduda@seas.upenn.edu}

% \color{FrontiersBlue} Is the blue color, used in the Journal name, in the title, and the names of the sections


\begin{document}
\onecolumn
\firstpage{1}

\title[Reproducibility of structural graph metrics]{Reproducibility of graph metrics of human brain structural networks}
\author[\firstAuthorLast ]{\Authors}
\address{}
\correspondance{}
\editor{}
\topic{Neuroinformatics with the Insight ToolKit}

\maketitle
\begin{abstract}

%\section{}
%As a primary goal, the abstract should render the general significance and conceptual advance of the work clearly accessible to a broad readership. References should not be cited in the abstract.
%See the Summary Table at \\ \url{http://www.frontiersin.org/}\texttt{\journal}\url{/authorguidelines} \\for abstract requirement and length according to article type.
% Abstract Max Length = 2000 characters
Recent interest in human brain connectivity has led to the application of
graph theoretical analysis to human brain structural networks, in
particular white matter connectivity inferred from diffusion imaging
and fiber tractography. While these methods have been used to study a
variety of patient populations, there has been less examination of the
reproducibility of these methods. These graph metrics typically derive
from fiber tractography, however a number of tractography algorithms
exist and many of these are known to be sensitive to user-selected
parameters. The methods used to derive a connectivity matrix from
fiber tractography output may also influence the resulting graph
metrics. Here we examine how these algorithm and parameter choices
influence the reproducibility of proposed graph metrics on a publicly
available test-retest dataset consisting of 21 healthy young
adults. Network summary
measures are examined using the intraclass correlation coefficient
(ICC), and the dice coefficient is used to examine overlap 
of constant density subgraphs. Functional data analysis techniques are used
to examine differences in network measures that result from the choice
of fiber tracking algorithm. The global and local efficiency measures were 
the most robust to the choice of fiber tracking algorithm.

%A freely available data set, the Multi-Modal MRI Reproducibility Resource, will serve as the basis for this study. ANTs, based upon ITKv4, will be used for the registration and segmentation steps needed to align and label individual brains. Camino, an open-source toolkit, will be used for: DT reconstruction, fiber tracking, and the generation of structural connectivity matrices. An ITK module will be created to implement the graph analysis metrics.

\tiny
  \section{Keywords:} Structure Tractography Connectivity Brain Network Reproducibility Graph  %All article types: you may provide up to 8 keywords; at least 5 are mandatory.
\end{abstract}


% For Technology Reports the introduction should be succinct, with no subheadings.
\section{Introduction}
Combining magnetic resonance imaging (MRI) of the human brain with graph
theory analysis has emerged as a powerful approach to studying
large-scale networks of both structural and functional
connectivity. In the case of structural connectivity, the use of diffusion weighted MRI and
associated fiber tractography methods provide the ability to identify
the long-range pathways that connect cortical regions and form a
network architecture~\citep{Basser2000,Lazar2003,Hagmann2003,Mori1999}. 
The use of graph theoretical analysis to study
the topology of these structural networks has increasingly been used to
examine the structural consequences of neurological disorders \citep{Xie2012,Basset2012}
as well as the relationship between structure and function \citep{}. 

Previous studies examining the reproducibly of graph-based metrics in
functional networks have shown good levels of reproducibly in MEG
\citep{Deuker2009}, fMRI using BOLD contrast
\citep{Telesford2010,Braun2012,Schwarz2011,Liang2012,Weber2013} and
arterial spin labeling \citep{Weber2013}. A number of studies have also examined
reproducibly in structural networks, each focusing on various aspects
of the complex processing pipeline that is a prerequisite for these
measures. These have included studies of diffusion spectrum imaging
\citep{Cammoun2012,Bassett2011N} and high angular resolution diffusion
imaging \citep{Dennis2012}. Some studies have examined probabilistic
tractography \citep{Owen2013BC,Vaessen2010}. DTI-based studies using
deterministic tractography have included the examination of
tractography seed density \citep{Cheng2012N}, anatomic label density
\citep{Bassett2011N}, and studies examining a variety of network
measures \citep{Cheng2012N,Irimia2012N}. 

In the paper we constrain our
analysis DTI-based deterministic fiber
tractography. Within this constraint, we examine multiple algorithms
for computing streamlines and their required parameters to examine
their influence on the final graph metrics. A set of manually defined
cortical parcellations \citep{Klein2012} is used along with a more common template-based
parcellation scheme \citep{AAL}. The intraclass correlation coefficient is used to 
examine the reproducibility of network summary measures that results from combinations
of fiber tracking algorithm and anatomical label set. Functional data analysis is used
to examine how these metrics differ as a function of graph density or other parameters
that are specific to a given metric. We use freely
available data and software to create a framework that facilitates future extensions
that may examine additional aspects of the processing as well as the
comparison to, or addition of, multiple imaging modalities. 

%Test retest of functional graph metrics via MEG \citep{Deuker2009}\\
%Test retest of functional graph metrics via fMRI \citep{Telesford2010,Liang2012,Braun2012}\\
%Test retest of structural graph metrics via DTI \citep{Owen2013BC}\\ 
%Test retest of structural graph metrics via DTI acquisition parameters \citep{Vaessen2010}\\
%Test retest of soft thresholding in fMRI \citep{Schwarz2011}\\
%Test restest of functional graph metrics via ASL \citep{Weber2013}\\
%Test retest of structural graph metrics via DTI and DSI with multiple labeling schemes \citep{Bassett2011N}\\
%Intra and inter subject variability of structural graph metrics via DTI for binary and weighted networks \citep{Cheng2012N}\\
%Correlations between pairs of regions using a variety of structural measures \citep{Irimia2012N}\\


%\textbf{Novel contributions}
%\begin{enumerate}
%\item Public data and fully open source
%\item In-depth examination of deterministic tractography parameters
%\item Probabilistic tractography extensions
%\item In-depth analysis of streamline-to-matrix conversion
%\item Provides plug-and-play framework for evaluation of new methods and/or alternate data sets
%\item Easy to extend to functional study (BOLD and ASL) 
%\end{enumerate}


%\begin{methods}
\section{Materials \& Methods}
% Materials and Methods: This section may be divided by subheadings. This section should contain sufficient detail so that when read in conjunction with cited references, all procedures can be repeated.

\subsection{Neuroimaging data}
The Multi-Modal MRI Reproducibility Resource \citep{Landman2011}
provides a publicly available test-retest data set consisting of 21
healthy control subjects (11 males). The mean
age is 31.76 $\pm$ 9.35 with a range of [22,61]. This data set provides a
multitude of MR image types, but here only the T1-weighted anatomical images and diffusion
tensor images are examined.

\subsection{Anatomical labeling}
The Mindboggle dataset provides a set of manually drawn cortical regions (DKT31) and a skull-stripped image
for a single time point for each subject \citep{Klein2012}. To utilize these labels in network creation we must perform an intrasubject
registration between T1 images and an inter modality registration from the T1 to DTI space for 
each time point. An affine registration is performed between T1 images. The brainmask is then warped into
the non-labeled time point. The Mindboggle brainmask include cerebellum and are applied to each T1 data set. The b=0
volume, acquired as part the diffusion tensor acquisition sequence is registered to the T1 image from the same acquisition. By composing
these transforms, the DKT31 labels are transferred into the DTI space for each time point using nearest neighbor interpolation. 

In addition to the DKT31 labels, the AAL label set was examined, due to it's prevalence in
the examination of both structural and functional connectivity~\citep{Tzourio-Mazoyer2002}. 
A similar registration-based approach is used to transform the AAL label in the DTI space.
An existing multivariate template had been created from the MMRR dataset using the
antsMultivariateTemplateConsturction.sh. Here we only use the T1 component of this
population-specific template. The AAL labels were transferred into the space of this template
via template-to-template registration using the antsRegistration tool. AAL labels that lie outside
of the cerebrum were excluded from later analysis. In order to transform these labels into each 
subject's T1 space, the antsCorticalThickness.sh tool was used. This software first applies bias
correction using the N4 algorithm. Next a registration based skull stripping is performed to provide
a cerebrum mask of the T1 image. This is followed by a final cerebrum-only registration to the template.
Composing the appropriate transforms again provides a method to transfer these labels in the DTI space for each time point. 
As the AAL labels include both gray and white matter, they are masked to only include voxels 
that have been identified as cortical gray matter by the DKT31 labels.

%ANTs volumetric-based cortical thickness estimation pipeline

%The N4 tool was used to perform bias correction on each subject's T1
%image \citep{Tustison????}. The antsRegistration tool was used to find
%a deformable mapping between each subject's T1 and the Mindboggle
%template for later use in anatomical labeling. An intra subject affine
%registration was used to align each subject's T1 images. Thresholding
%and a morphological closing was
%used to obtain a brainmask from the brain extracted T1 images provided
%by Mindboggle. 

%For each subject, a set of manually defined cortical labels was
%available via Mindboggle \citep{Klein2012}. Additionally, the AAL label set
%\citep{Tzourio-Mazoyer2002} was also examined
%as it is a label set often used in both functional and structural
%studies. Here a template-based approach is used and the
%antsRegistration tool is used to find a deformable mapping to the
%Mindboggle template. These template mapping are then combined with the
%intrasubject mapping to transfer the AAL labels into the DTI space for
%each time point.

\subsection{Diffusion data preprocessing}
The Camino toolkit \citep{} is used to calculate diffusion tensor images via a
weighted linear fitting \citep{Basser1994,Salvador2005}, and is also
used for subsequent deterministic tractography. The brainmasks defined
in anatomical space are warped into DTI space and are used to restrict the
tractography to eliminate false positives that could results from
streamlines that leave and then reenter the brain. Fractional anisotropy (FA) images are calculated and a
tractography seed-map is created by including all voxels in the cerebrum with an FA of at
least 0.2. 

%These seed-maps are then resampled to a resolution of 0.5mm x 0.5mm x
%0.5mm to provide a dense seeding as this has been shown to increase stability of
%structural network metrics ~\citep{Cheng2012}. 

One of the primary differences among the various approaches to
deterministic tractography is the algorithm used to determine the
direction that a streamline should proceed from a given point. Here we
examine four different approaches:

\begin{enumerate}
\item Fiber Assignment by Continuous Tracking (FACT) - The primary
  direction of diffusion (PDD) is followed until the streamline enters
  a new voxel \citep{Mori1999}.
\item Euler -  The PDD is followed for a constant step size \citep{Basser2000}.
\item Rourth-order Runge-Kutta (RK4) - The direction of the step is determined
 by taking and averaging a weighted series of partial steps \citep{Basser2000}.
\item Tensor Deflection (TEND) - The entire diffusion tensor is used to deflect
the estimated fiber trajectory \citep{Lazar2003}
\end{enumerate}

Shared parameters used in the fiber tracking were held constant as follows
\begin{enumerate}
\item Streamlines were terminated if curvature of more than 90 degrees over 5 steps was detected. 
\item Streamlines were terminated in an FA value of less of 0.2 was encountered. 
\item A step size of 0.5mm was used. 
\item Linear interpolation of the primary direction of diffusion was used for Euler and RK4. 
\end{enumerate}

%Additionally we examine the weighting parameters used in the TEND
%algorithm

\subsection{Graph generation}
For a given set of streamlines, the connmat tool provided by  the Camino toolkit was used to generate
a connectivity matrix that records how many streamlines connect each pair of
regions in a given set of target regions. This program starts at the seed point for a streamline and proceeds outward in each
direction to determines the two target regions encountered. Only streamlines that connect two unique regions are retained and
a given streamline may be only be counted as connecting a single pair of target regions. 

To compare graphs it is necessary to first ensure that the graphs are of equal density, where density for an undirected graph is defined as:
$$D(G) = \frac{\|E(G)\|}{( \|N(G)\| (\|N(G)\|-1) )} $$
where $N(G)$ is the set of all nodes in graph $G$ and $E(G)$ is the set of all edges in $G$. To obtain constant density graphs, the $M$ highest weighted edges are retained and all other edges are removed where $M$ is fully determined by the desired density and number of nodes in the graph. This cumulative thresholding provides a normalized method for comparing network measures as it results in the comparison of graphs with an equal percentage of significant connections. Here, we only directly compare measures obtained from graphs with an equal number of nodes.

%Additionally, we generative matrices
%that record the average FA along all tracts that connect a given pair
%of regions. In both cases, the matrices may be thresholded for
%unweighted graph metrics or used as is for weighted graph metrics. 

%\begin{table}[!t]
%\processtable{Descriptions and references for graph metrics examined in this study.\label{tab:nodes}}
%{\begin{tabular}{lll}
%\midrule
%Node metrics & Description & Reference\\\midrule
%Degree & Number of connections for a node & \\
%Clustering coefficient & Local neighborhood connectivity & \citep{Watts1998}\\
%Path length & Average shortest path to all other nodes & \citep{Watts1998}\\
%Global efficiency & ``Closeness'' to all other nodes & \citep{Latora2001}\\
%Local efficiency & ``Closeness'' to local nodes & \\
%\midrule
%Whole-graph metrics\\
%\midrule
%Small-world & & \citep{Watts1998}\\
%Synchronizability & & \citep{Motter2005}\\
%Assortativity & & \citep{Newman2002}\\
%Hierarchy & & \citep{Ravasz2003}\\
%Cost efficiency & & \citep{Achard2007}\\
%Rich-club coefficient & Degree to which high-degree nodes preferentially inter-connect & \citep{Colizza2006}\\
%\midrule
%Network similarity measures\\
%\midrule
%Network overlap & & \citep{vanWijk2010}\\
%Edge overlap & Percentage of common edges in constant density networks & \citep{Weber2013} \\\botrule
%\end{tabular}}{}
%\end{table}

\subsection{Network metrics}
%As we are interested in whole network summary measures, we examine the mean
%over all nodes for each of the node-metrics listed in
%table \label{tab:nodes}. For more details on the node metrics see
A large number of metric are available for quantifying properties of binary, undirected networks \citep{Rubinov2010}. Here we examine a number that 
are common in current literature: largest connected component size \citep{Basset2011}, assortativity\citep{Newwan2006,Basset2008}, clustering coefficient \citep{Watts1998}, characteristic path length\citep{Watts1998}, global and local efficiency \citep{Latora2001}, and rich club coefficient\citep{RichClub}. An ITK module named Petiole (https://github.com/jeffduda/Petiole) was created to calculate these network measures from 2D connectivity
matrices \citep{Petiole}. This module incorporates and extends an existing implementation of a graph class \citep{NickITKJournal} and provides ITK functions for a variety of graph metrics while using the matlab-based Brain Connectivity Toolkit \citep{BCT} for algorithmic guidance. While many of these metrics include implementations for weighted graphs and/or directed graphs, here we focus on their application to unweighted, undirected graphs. Summaries and equations for these metrics are provided here:

\emph{Size of Largest Connected Component}.  A connected component of a graph is a subset of the graph, $G_{i}$, where there exists a path between all pairs of nodes and for which no path exist to additional nodes in $G$. The largest connected component is the $G_{i}$ with the greatest number of nodes, $\|N(G_{i}\|$. This measure relates to the global level of connectivity within a subject's brain network~\cite{Basset2011}.


\emph{Assortativity}. Assortativity measures how preferentially nodes of similar degree connect to one another \citep{Newman2006} and is defined as:
$$A =  \frac{ \frac{1}{E} \sum_{i}{j_i k_i} - [ \frac{1}{E} \sum_{i}{ \frac{1}{2} (j_i + k_i)} ]^2 }{ \frac{1}{E} \sum_{i}{ \frac{1}{2}( j_{i}^{2} + k_{i}^{2} ) } - [ \frac{1}{E} \sum_{i}{ \frac{1}{2} (j_i + k_i)} ]^2  } $$
where $j_i,k_i$ are the degrees of the nodes connected by edge $i$ and $E = \|E(G)\|$.

\emph{Clustering Coefficient}. This measure quantifies how likely is that two nodes with a common neighbor are connected to one another ~\citep{Watts1998}. Here we calculate the clustering coefficient at each node and calculate the mean over all nodes in the network for our final network summary measure. The clustering coefficient at node $i$ is given by:
$$C_i = \frac{e_i}{\|K_i\| ( \|K_i\| -1 )}$$
where $K_i$ is the set of all nodes that share an edge with $i$ and $e_i$ is the set of all edges that connect nodes in $K_i$.

\emph{Characteristic Path Length}. The pathlength, $L_{ij}$, that connects two nodes, $i$ and $j$, is defined as the minimum number of edges that must be traversed to travel from $i$ to $j$ \citep{Dijkstra1959}. The characteristic path length is the average pathlength over all possible pairs of connections in a graph. In an undirected graph this is:
$$L = \frac{1}{\|N(G)\|(\|N(G)\|-1)} \sum_{ij \in G, i \neq j}{L_{ij}}$$
This measure is only defined for fully connected graphs. Here, we apply the density thresholding first and then extract the largest connected component in order to calculate the characteristic path length.

\emph{Global Efficiency}. This measure is related to the characteristic path length, in that it attempts to quantify the mean efficiency between any two nodes in the graph. However, this metric is defined for both connected unconnected graphs \citep{Latora2001}.
$$F_{glob} = \frac{1}{\|N(G)\|(\|N(G)\|-1)} \sum_{i \neq j \in G}{1/L_{ij}}$$


\emph{Local Efficiency}. This metric relates to fault tolerance and examines efficiency between neighbors on a node $i$, if that node were removed from the graph \citep{Latora2001}.
$$F_{loc} = \frac{1}{\|N(G)\|} \sum_{i \in n}{F(G_i)}$$
where $G_i$ is the subgraph of $G$ that results from removing node $i$.

\emph{Rich Club Coefficient}
This measures quantifies how preferentially the high-degree nodes (i.e. rich nodes)  in a graph connect to other high-degree nodes \citep{RichClub}.
$$ R(G,k) = \frac{ \|E(G,k)\| }{ \|N(G,k)\| ( \|N(G,k)\|-1 ) } $$
where $N(G,k)$ is the set of nodes of degree k or higher and $E(G,k)$ is the set of edges connecting two nodes in $N(G,k)$.


\subsection{Graph curves}
The metrics listed above are all applied to thresholded binary graphs. As discussed earlier, this binary graph result from thresholding at a constant density. These metrics may then be treated as functional curves of metric vs. graph density. By doing this, we are able to compare binary graphs in a way that incorporates the continuous structure of the original connectivity matrices. The rich club coefficient however is dependent upon two parameters, the graph density and, $k$, the degree threshold used to determined what constitutes a rich-node. For this metric we threshold at the highest density common to all graphs and explore how the value changes with $k$. For all other metrics, we examine their curves as a function of graph density. 

\subsection{Statistical analysis}
Before examining how the graph metrics change with density it is necessary to examine the maximum density of the graphs to determine the range over which graph curves may be compared. Additionally, it is interesting to examine the topological similarity in the thresholded graphs. This is done using the dice coefficient which measures similarity between two graphs as:
$$Dice(x,y) = \frac{ 2 \| E(x) \cap E(y) \| }{ \|E(x) \| + \| E(y) |\ }$$
where edges are considered equal if they connect the same two nodes. The mean intra- and inter-subject topological similarity 
was computed over a range of densities for each combination of tracking algorithm and anatomical label sets. This allows us to 
examine the reproducibility of within-subject topography compared to between subject topography. This metric is limited to lie in the range $[0,1]$ and can be interpreted as a measure of degree of overlap between graphs. This provides a stricter metric than measuring overlap between sets of nodes as complete node-overlap is a necessary but incomplete condition for complete edge-overlap.

Graph curves are used to examine the reproducibly of the graph metrics as a function of an independent parameter, typically graph density. At each point along the curve, reproducibility of the metric is quantified using the ICC:
$$ICC = \frac{\sigma_{bs}^{2}}{\sigma_{bs}^{2} + \sigma_{ws}^{2}} $$
where $\sigma_{bs}^{2}$ is the between-subject variance and $\sigma_{ws}^{2}$ is the within subject variance. The 'ICC' package for  \R{} is used for this calculation. The ICC is plotted along with the mean graph metrics for each combination of algorithm and label set. At points where little to no variance exists in a graph metric, the ICC is not calculated as it becomes unstable under those conditions. The following guidelines may be used to interpret ICC values: ICC $<$ 0.2 'poor agreement'; 0.21 - 0.40 'fair agreement'; 0.41-0.60 moderate agreement; 0.61-0.80 'strong agreement'; ICC $>$ 0.8 'near perfect agrement \citep{Telesford2010,Montgomery2002}. Dashed lines indicating the boundaries of these categories have been included on all ICC plots to aid interpretation.

To identify group diferences that result from fiber tracking algorithm we incorporated methods from functional data analysis which treats each curve as a function. While there are a variety of methods for computing the difference between two curves, here we choose the simplest method, the non-parametric permutation test. We first treat each mean curve as a function and find the area between a pair of curves. We then permute group assignment of individuals used to calculate the mean curves using random sampling without replacement. We then calculate new mean curves from the randomly assigned groups and measure the area between these curves. This it performed iteratively (i=100000). We record, x, the number of times area between the mean curves from the randomly assigned groups is larger than the area between the true group mean curves. The p-value for the true group difference is then defined as $x/i$. We report these differences for between-algorithm curves as they derive from graph of equal size, but do compare curves that derive from different anatomical label sets.

%\textbf{Figure 1.}{ Enter the caption for your figure here.  Repeat as  necessary for each of your figures.}\label{fig:01}% Don't add the figures in the LaTeX files, please upload them when submitting the article. Frontiers will add the figures at the end of the provisional pdf.

%\end{methods}

\section{Results}
%Results: This section may be divided by subheadings. Footnotes should not be used and have to be transferred into the main text
\subsection{Network density}
Maximal densities for connectivity matrices across all tracking algorithm ranged from 0.17 to 0.30 for the AAL labels and from 0.20 to 0.41 for DKT31. Maximal densities in the DTK31 data was generally higher than in the AAL as illustrated in figure \ref{fig:density}. Both label sets had the same lowest-to-highest ordering of mean maximal density within algorithms: RK4 $<$ Euler $<$ FACT $<$ TEND.

\begin{figure}
\begin{center}
\includegraphics[width=0.5\linewidth]{figures/density_plot.png} 
\caption{Boxplots illustrating the density values for unthresholded connectivity matrices for all subjects and all time points, grouped by fiber tracking algorithm (Euler,FACT,RK4,TEND) and anatomical label set (AAL,DKT31).}
\label{fig:density}
\end{center}
\end{figure}

\subsection{Network topology}
Dice coefficients for intra-subject similarity ranged from 0.70 to 0.81 for the AAL labels and from 0.59 to 0.82 for the DTK31 labels. Inter-subject similarity ranged from 0.51 to 0.71 for AAL labels and from 0.32 to 0.71 for the DTK31 labels. For all algorithm-label pairings, intra-subject overlap was greater than inter-subject overlap across the range of densities as illustrated in figure \ref{fig:dice}. Permutation testing of intra-subject dice vs. density curves did not reveal any significant differences between algorithms for either label set. However, a number of differences were found in the inter-subject comparions. The resulting p-values are listed in table \ref{tab:dicep}.

\begin{figure}
\begin{center}
\includegraphics[width=0.5\linewidth]{figures/dice_overlap_plot.png} 
\caption{Connectivity matrices were thresholded over a range of density values. At each density level, consistency of network topography was estimated by calculating the mean dice overlap for both intra subject and intersubject pairs.}
\label{fig:dice}
\end{center}
\end{figure}

%\begin{table}[!t]
%\processtable{Dice p-values.\label{tab:dicep}}
%{\begin{tabular}{l | llll}
%\midrule
 %        & Euler    & FACT     & RK4        & TEND      \\ \midrule
%Euler  &            & 0.9555  & 0.9864    & 0.6549   \\
%FACT & 0.7770 &              & 0.7970  & 0.9524     \\
%RK4   & 0.2675 & 0.2586   &               & 0.5358    \\
%TEND & 0.0351 & 0.0351*  & 0.1335    &                 &    
%\end{tabular}}{}
%\end{table}

\begin{table*}[!t]
\processtable{Functional data analysis is used along with permutation
  testing to look for differences in dice overlap measures between graph 
generated from different fiber tracking algorithms. Upper triangular values
are for the AAL labels, while lower triangular are for the DTK31 label set.\label{tab:dicep}}
{
\begin{tabular}{l | llll | llll }
\midrule
 & \multicolumn{4}{c}{Intra Subject}  & \multicolumn{4}{c}{Inter Subject} \\ 
\midrule
Euler  &            & 0.9555  & 0.9864    & 0.6549   &              & 0.7770  & 0.2675 & 0.0351*  \\
FACT & 0.4186 &              & 0.7970  & 0.9524     & 0.0002*  &             & 0.2586 & 0.0355* \\
RK4   & 0.8780 & 0.6179   &               & 0.5358   & 0.0632   & 0.0014* &            & 0.1335   \\
TEND & 0.3952 & 0.6655  & 0.7564    &            &    0.0001*  & 0.0003* & 0.0838 &               \\             
\midrule
         & Euler    & FACT     & RK4        & TEND      & Euler     & FACT     & RK4     & TEND       \\
\midrule
\end{tabular}}{}
\end{table*}


\subsection{Network summary measures over graph density}
For each combination of tracking and label set, the mean curves that were calculated to examine how the metrics change as a 
function of graph density are illustrated in figure \ref{fig:vsdensity} along with the ICC curves that quantify reproducibility. While
no direct comparisons were done between graph curves generated from different label sets, the scales of the plots are equal to 
allow for convenient visual comparison. Only the characteristic path length curves exhibit a different shape between label sets, and
only at low density values. This is likely a results of the smaller number of regions in DKT31 label set. Clustering coefficient, and
global and local efficiency exhibit the most similarity across label sets. Comparing within metric and within label set, the fiber tracking
algorithms appear consistent as far as shape. Functional data analysis, along with permutation testing does reveal a number of significant
differences between graph curves however, as listed in table \ref{tab:permtesting}. No significant differences were found between 
tracking algorithms using the DKT31 labels. Within the AAL labels, significant differences were found between RK4 and TEND for four 
of the six metrics examined.

\begin{figure*}
\begin{center}
\begin{tabular}{cc}
\includegraphics[width=0.5\linewidth]{figures/clust_plot.png}  & \includegraphics[width=0.5\linewidth]{figures/path_plot.png} \\
\includegraphics[width=0.5\linewidth]{figures/clust_icc_plot.png} & \includegraphics[width=0.5\linewidth]{figures/path_icc_plot.png} \\
A & B \\
\includegraphics[width=0.5\linewidth]{figures/size_plot.png}  & \includegraphics[width=0.5\linewidth]{figures/assort_plot.png} \\
\includegraphics[width=0.5\linewidth]{figures/size_icc_plot.png} & \includegraphics[width=0.5\linewidth]{figures/assort_icc_plot.png} \\
C & D \\
\includegraphics[width=0.5\linewidth]{figures/geff_plot.png}  & \includegraphics[width=0.5\linewidth]{figures/leff_plot.png} \\
\includegraphics[width=0.5\linewidth]{figures/geff_icc_plot.png} & \includegraphics[width=0.5\linewidth]{figures/leff_icc_plot.png}\\
E & F 
\end{tabular}
\caption{Graph metric vs. graph density plots along with corresponding ICC plots for A) Mean clustering coefficient  B) Characteristic path length C) Largest connected component size D) Assortativity E) Global efficiency and F) Local efficiency}
\label{fig:vsdensity}
\end{center}
\end{figure*}


\begin{table*}[!t]
\processtable{Functional data analysis is used along with permutation
  testing to look for pair-wise differences in graph-metric vs. graph-density curves that result from different
  fiber tracking algorithms and label sets. Only the first time-point for each
  subject is used. For each metric, the upper-triangular values are for p-values for
  the AAL labels while the lower-triangular values were generated with
  the DKT31 label set.\label{tab:permtesting}}
{\begin{tabular}{l | llll | llll }
\midrule
 & \multicolumn{4}{c}{Clustering Coefficient} &  \multicolumn{4}{c}{Characteristic Path Length} \\ 
\midrule
Euler  &            & 0.2164  & 0.5296    & 0.0346*   &            & 0.2786  & 0.2389    &  0.4728  \\
FACT & 0.6962 &              & 0.0246*  & 0.2822     & 0.9982 &            &  0.0145*   & 0.1235     \\
RK4   & 0.9927 & 0.7958  &               & 0.0049*   & 0.8465 & 0.9199  &                 & 0.3031   \\
TEND & 0.1327 & 0.8858  & 0.2061    &                & 0.4854 & 0.6459  & 0.8234   &    \\ 
\midrule
 & \multicolumn{4}{c}{Connected Component Size} &  \multicolumn{4}{c}{Assortativity} \\ 
\midrule
Euler  &            & 0.3471  & 0.9324    & 0.7556   &            & 0.3680  & 0.3294   &  0.4651  \\
FACT & 0.9447 &              & 0.0468*  & 0.3025     & 0.6361 &            &  0.0270*   & 0.8877     \\
RK4   & 0.9998 &  0.7610  &              & 0.4748   & 0.9326 & 0.3250  &               & 0.0666   \\
TEND & 0.7912 & 0.8336  & 0.7269   &                & 0.8272 & 0.4021  & 0.9895   &    \\ 
\midrule
 & \multicolumn{4}{c}{Global Efficiency} &  \multicolumn{4}{c}{Local efficiency} \\ 
\midrule
Euler  &            & 0.8617  & 0.9861    & 0.7272   &            & 0.4579  & 0.9227   &  0.6065  \\
FACT & 0.8677 &              & 0.6882  & 0.7667     & 0.8794 &            &  0.2486   & 0.1230     \\
RK4   & 0.6295 &  0.9415  &              & 0.8677   & 0.9745 & 0.4413  &               & 0.7557   \\
TEND & 0.9977 & 0.9633  & 0.7438    &                & 0.8752 & 0.7987  & 0.4775   &    \\ 
\midrule
         & Euler    & FACT     & RK4        & TEND       & Euler    & FACT     & RK4        & TEND      \\
\midrule
\end{tabular}}{}
\end{table*}

\subsection{Rich club coefficient over node-degree}
Because the rich club coefficient requires the selection of multiple parameters, we chose to examine how this metric
changes as a function of $k$, the node degree that determines what is considered a 'rich' node. The plots for the
mean graph curves and ICC coefficients are illustrated in figure \ref{fig:richclub}. The results are similar to the examinations
over graph density in that the same shape appears for both label sets, but with an scaling difference and the tracking algorithms
have similar shapes but within the AAL networks, differences were found in the RK4-FACT (p$=$0.0338) and RK4-TEND (p$=$0.0252) 
comparisons. The p-values for all comparisons are listed in table \ref{tab:richp}

\begin{figure}
\begin{center}
\includegraphics[width=0.5\linewidth]{figures/richclub_plot.png} \\
\includegraphics[width=0.5\linewidth]{figures/richclub_icc_plot.png}
\caption{Rich club coefficient was examined over a range of levels, k, and a constant graph density of 0.15}
\label{fig:richclub}
\end{center}
\end{figure}

\begin{table*}[!t]
\processtable{Functional data analysis is used along with permutation
  testing to look for differences in rich club coefficients  
generated from different fiber tracking algorithms. Upper triangular values
are for the AAL labels, while lower triangular are for the DTK31 label set.\label{tab:richp}}
{
\begin{tabular}{l | llll }
\midrule
Euler  &            & 0.3359  & 0.6830    & 0.2064   \\
FACT & 0.8944 &              & 0.0338*  & 0.8537     \\
RK4   & 0.9282 & 0.7219   &               & 0.0252*   \\
TEND & 0.7548 & 0.8800  & 0.3360    &                \\             
\midrule
         & Euler    & FACT     & RK4        & TEND        \\
\midrule
\end{tabular}}{}
\end{table*}



%\begin{table}[!t]
%\processtable{Resolution Requirements for the figures\label{Tab:01}}
%{\begin{tabular}{lllll}\toprule
%Image Type & Description & Format & Color Mode & Resolution\\\midrule
%Line Art & An image composed of lines and text,  & TIFF, EPS, JPEG & RGB, Bitmap & 900 - 1200 dpi\\
%          & which does not contain tonal or shaded areas.& & &\\
%         Halftone & A continuous tone photograph, which contains no text. & TIFF, EPS, JPEG & RGB, Grayscale & 300 dpi\\
%Combination & Image contains halftone + text or line art elements. & TIFF, EPS, JPEG & RGB,Grayscale & 600 - 900 dpi\\\botrule
%\end{tabular}}{This is a footnote}
%\end{table}

%\begin{equation}
%\sum x+ y =Z\label{eq:01}
%\end{equation}

%\textbf{Table\ref{Tab:01}} shows the resolution requirements for the figures. The figures must be legible:
%\begin{enumerate}
%\item The smallest visible text is no less than 8 points in height, when viewed at actual size.
%\item Solid lines are not broken up.
%\item Image areas are not pixelated or stair stepped.
%\item Text is legible and of high quality.
%\item Any lines in the graphic are no smaller than 2 points width.
%\end{enumerate}

%\textbf{Figure 2.}{ Enter the caption for your figure here.  Repeat as  necessary for each of your figures.}\label{fig:02}

\section{Discussion}
% Discussion: This section may be divided by subheadings. Discussions
% should cover the key findings of the study: discuss any prior art
% related to the subject so to place the novelty of the discovery in
% the appropriate context; discuss the potential short-comings and
% limitations on their interpretations; discuss their integration into
% the current understanding of the problem and how this advances the
% current views; speculate on the future direction of the research and
% freely postulate theories that could be tested in the future.


%Other modalities, not examined here,
%were also acquired making this data useful for future examinations of
%structure and function.

%No smoothing of data here

%Other DTI scalar metrics, such as RD, or from other modalities such as MTR.
%%
%Did not normalize matrices

\subsection{Network topology}
Although a number of studies have examined the reproducibility of graph metrics on 
structural brain networks derived from DTI-based fiber tractography, there are no known 
papers that focus on the selection of deterministic tracking algorithm. To facilitate later examination of
graph metrics as a function of graph density, we first examined the reliability of identifying
subgraphs by thresholding. Using the dice coefficient as a measure of overlap we demonstrated
that the intra subject agreement was much higher than the inter subject agreement across all tracking algorithms and 
label sets. Using the AAL label set we demonstrated differences in topology between FACT and TEND which suggest
that the TEND algorithm may provide a more reproducible measure of network topology across subjects. This is significant as the AAL
labels are prevalent throughout all forms of connectivity studies and the FACT algorithm is the most common choice
among deterministic tractography methods. However, further analysis of additional tracking parameters is 
necessary to determine the full set of conditions under which this difference holds.

\subsection{Network summary measures over graph density}
Global and local efficiency are the most robust to choice of fiber tracking algorithm, and have high levels 
of reprocibily across density levels. Assortativity and characteristic path length are highly repoducible across
density levels, but are sensitive to choice of fiber tracking algorithm. In general, the portions of the graph curves
at low density value and less reproducible than the segments at high density. 

\subsection{Rich club coefficient over node-degree}
The examination of rich club coefficient as a function of degree-level demonstrates the use of graph curves over 
a parameter other than graph density. Consistency appears to have a somewhat inverse relationship to the x
coefficient as a function of node degree level. This is a result of the fact that the rich club coefficient values converge
at high and low densitites. Here, all graphs were thresholded at the maximum density achievable by all graphs. As the average node degree would drop with lowered
density, additional work is required to more thoroughly understand the relationship between graph-density and degree-level that 
would provide the most reproducible results. 

\subsection{Limitations and future directions}
The are a number of methodological limitations to the work presented here. We limited the fiber tracking to deterministic
methods and used constant shared parameters for these methods. The influence of these parameters on individual 
tracking algorithms and the resulting graph metrics demands further exploration. In the choice of anatomical label sets, we
limited the analysis to a set of manually defined labels, and a often used set of template-based labels. In each case we used the label
'as-is' without upsampling to a higher number of regions. 

An additional limitation of this work is the use of streamline count matrices as the basis for thresholding to create constant
density graphs. Multiple options exist for normalizing the streamline count matrices using the volumes of 
the target cortical regions and/or the average length of the streamlines the connect two regions. The volume based
normalization may accomodate the differences that are seen between graph curves that were generated using 
the different anatomical labels. However, the focus here was on the influence of the fiber tracking and no direct
comparisons were made between graph curves generated from the different label sets. A number of additional options
exist for creating a weighted connectivity matrix inluding the averge FA of fibers that connect two regions. Since the data set
examined also includes magenetization transfer data, the average magnetization transfer ratio along streamlines 
could potentially be useful as it directly related to myelin content in white matter. These issues were beyond the scope 
of the current study but would made for an intriguing extension of the current work. 

The selection of graph metrics for analysis is another limitation of the study. An exhaustive examination of all
possible graph metrics was not feasible so metrics that have been studied previously were chosen to give additional
context to existing work. Many of the metrics examined have alternate formulations for 
weighted graphs. Here, only unweighted graph metrics were examined as they are prevelant in current literature.
 The creation of a testing framework that relies upon a public data set and open-source code
was intended to facilitate the further exploration of the issues listed here.

\subsection{Conclusion}
This study evaluate the reproducibility of graph summary metrics in structural brain networks derived from
DTI based deterministic fiber tractography. Four differenent fiber tracking algorithms were examined along
with two different anatomical label set. A number of graph metrics were examined by creating graph curves that capture how a metric
changes over a parameter such as graph density. ICC plots were used to evaluate the reprociblity of the metrics and FDA
was used to indentify significant differences between graph curves generated using different fiber tracking algorithms.
While differenences between the tracking algorithms were not drastic, they were significant in many cases, suggesting
that future studies should give careful consideration to the choice of fiber tracking algorithm based upon the graph
metric that will be analyzed. 


\subsection{Data Sharing}
%Frontiers supports the policy of data sharing, and authors are advised to make freely available any materials and information described in their article, and any data relevant to the article (while not compromising confidentiality in the context of human-subject research) that may be reasonably requested by others for the purpose of academic and non-commercial research. In regards to deposition of data and data sharing through databases, Frontiers urges authors to comply with the current best practices within their discipline.
Free, publicly-available data and software was used throughout. The scripts used to generate the data and figures are available at: https://github.com/jeffduda/StructConnRepro. This repository contains the configuration file that, when added to ITK, will download and compile Petiole which builds the executables that were used to generate the graph metrics examined in this study. Also, the final segmentations used as the target regions for fiber tracking are included in this repository to provide a convenient starting point for reproducing or extending the methods presented here.

\section*{Disclosure/Conflict-of-Interest Statement}
%All relationships financial, commercial or otherwise that might be perceived by the academic community as representing a potential conflict of interest must be described. If no such relationship exists, authors will be asked to declare that the research was conducted in the absence of any commercial or financial relationships that could be construed as a potential conflict of interest.
The authors declare that the research was conducted in the absence of any commercial or financial relationships that could be construed as a potential conflict of interest.

%\section*{Acknowledgement}
%Shoutouts to our peeps

%\paragraph{Funding\textcolon} Shoutout to our peep\$

%\section*{Supplemental Data}
%Maybe need this, maybe not

\bibliographystyle{frontiersinSCNS} % for Science articles
%\bibliographystyle{frontiersinMED} % for Medicine articles
\bibliography{priorwork}

%\begin{thebibliography}{}

%\bibitem[Bofelli {et~al}., 2000]{Boffelli03} Bofelli,F., Name2, Name3 (2003) Article title, {\it Journal Name}, 199, 133-154.

%\bibitem[Bag {et~al}., 2001]{Bag01} Bag,M., Name2, Name3 (2001) Article title, {\it Journal Name}, 99, 33-54.

%\end{thebibliography}
\end{document}
