%%%%%%%%%%%%%%%%%%%%%%%%%%%%%%%%%%%%%%%%%%%%%%%%%%%%%%%%%%%%%%%%%%%%%%%%%%%%%%%%%%%%%%%%%%%%%%%%%%%%%%%%%%%%%%%%%%%%%%%%%%%%%%%%%%%%%%%%
% This is just a template to use when submitting manuscripts to Frontiers, it is not mandatory to use frontiers.cls nor frontiers.tex  %
%%%%%%%%%%%%%%%%%%%%%%%%%%%%%%%%%%%%%%%%%%%%%%%%%%%%%%%%%%%%%%%%%%%%%%%%%%%%%%%%%%%%%%%%%%%%%%%%%%%%%%%%%%%%%%%%%%%%%%%%%%%%%%%%%%%%%%%%

\documentclass{frontiersSCNS} % for Science articles
%\documentclass{frontiersMED} % for Medicine articles

\usepackage{url}
\usepackage{tabularx}
\usepackage{lineno}
\usepackage[english]{babel}
\linenumbers


\copyrightyear{}
\pubyear{}
%\onecolumn
%%% write here for which journal %%%
\def\journal{Neurosciences}
\def\DOI{}
\def\articleType{}
\def\citing{\color{darkgray}\cite}
\def\keyFont{\fontsize{6}{11}\helveticabold }
\def\firstAuthorLast{Duda {et~al}} %use et al only if is more than 1 author
\def\Authors{Jeffrey T. Duda\,$^{1,*}$, Philip A. Cook\,$^{1}$ and James C. Gee\,$^1$}
% Affiliations should be keyed to the author's name with superscript numbers and be listed as follows: Laboratory, Institute, Department, Organization, City, State abbreviation (USA, Canada, Australia), and Country (without detailed address information such as city zip codes or street names).
% If one of the authors has a change of address, list the new address below the correspondence details using a superscript symbol and use the same symbol to indicate the author in the author list.
\def\Address{$^{1}$Penn Image Computing and Science Laboratory, University of Pennsylvania, Department of Radiology, Philadelphia, PA, USA}
% The Corresponding Author should be marked with an asterisk
% Provide the exact contact address (this time including street name and city zip code) and email of the corresponding author
\def\corrAuthor{Jeffrey T. Duda}
\def\corrAddress{Penn Image Computing and Science Laboratory, University of Pennsylvania, Department of Radiology, 3600 Market Street, Suite 370, Philadelphia, PA, USA}
\def\corrEmail{jtduda@seas.upenn.edu}

% \color{FrontiersBlue} Is the blue color, used in the Journal name, in the title, and the names of the sections


\begin{document}
\onecolumn
\firstpage{1}

\title[Reproducibility of structural graph metrics]{Reproducibility of graph metrics of human brain structural networks}
\author[\firstAuthorLast ]{\Authors}
\address{}
\correspondance{}
\editor{}
\topic{Neuroinformatics with the Insight ToolKit}

\maketitle
\begin{abstract}

\section{}
%As a primary goal, the abstract should render the general significance and conceptual advance of the work clearly accessible to a broad readership. References should not be cited in the abstract.
%See the Summary Table at \\ \url{http://www.frontiersin.org/}\texttt{\journal}\url{/authorguidelines} \\for abstract requirement and length according to article type.
% Abstract Max Length = 2000 characters
% This was proposed abstract, change to legit abstract
Recent interest in the human connectome has led to the application of graph theoretical analysis to human brain structural networks, in particular white matter connectivity inferred from diffusion imaging and fiber tractography. While these methods have been used to study a variety of patient populations, there has been less examination of the reproducibility of these methods. These graph metrics typically derive from fiber tractography, however a number of tractography algorithms exist and many of these are known to be sensitive to user-selected parameters. The methods used to derive a connectivity matrix from fiber tractography output also influence the resulting graph metrics. Here we examine how these algorithm and parameter choices influence the reproducibility of proposed graph metrics. 

%A freely available data set, the Multi-Modal MRI Reproducibility Resource, will serve as the basis for this study. ANTs, based upon ITKv4, will be used for the registration and segmentation steps needed to align and label individual brains. Camino, an open-source toolkit, will be used for: DT reconstruction, fiber tracking, and the generation of structural connectivity matrices. An ITK module will be created to implement the graph analysis metrics.


\tiny
  \section{Keywords:} Structure Tractography Connectivity Brain Network Reproducibility  %All article types: you may provide up to 8 keywords; at least 5 are mandatory.
\end{abstract}


\section{Introduction}

% For Technology Reports the introduction should be succinct, with no subheadings.
Test retest of functional graph metrics via MEG \citep{Deuker2009}\\
Test retest of functional graph metrics via fMRI \citep{Telesford2010}\\
Test retest of structural graph metrics via DTI \citep{Owen2013BC}\\ 
Test retest of structural graph metrics via DTI and DSI with multiple labeling schemes \citep{Bassett2011N}\\
Intra and inter subject variability of structural graph metrics via DTI for binary and weighted networks \citep{Cheng2012N}\\
Correlations between pairs of regions using a variety of structural measures \citep{Irimia2012N}\\

\textbf{Novel contributions}
\begin{enumerate}
\item Public data and fully open source
\item In-depth examination of deterministic tractography parameters
\item Probabilistic tractography extensions
\item In-depth analysis of streamline-to-matrix conversion
\item Provides plug-and-play framework for evaluation of new methods and/or alternate data sets
\item Easy to extend to functional study (BOLD and ASL) 
\end{enumerate}


%\begin{methods}
\section{Material \& Methods}
% Materials and Methods: This section may be divided by subheadings. This section should contain sufficient detail so that when read in conjunction with cited references, all procedures can be repeated.
Provide an overview of what we are examining here

\begin{table}[!t]
\processtable{Descriptions and references for graph metrics examined in this study.\label{Tab:01}}
{\begin{tabular}{lll}
\midrule
Node metrics & Description & Reference\\\midrule
Degree & Number of connections for a node & \\
Clustering coefficient & Local neighborhood connectivity & \citep{Watts1998}\\
Path length & Average shortest path to all other nodes & \citep{Watts1998}\\
Global efficiency & ``Closeness'' to all other nodes & \citep{Latora2001}\\
Local efficiency & ``Closeness'' to local nodes & \\
\midrule
Whole-graph metrics\\
\midrule
Small-world & & \citep{Watts1998}\\
Synchronizability & & \citep{Motter2005}\\
Assortativity & & \citep{Newman2002}\\
Hierarchy & & \citep{Ravasz2003}\\
Cost efficiency & & \citep{Achard2007}\\
Rich-club coefficient & Degree to which high-degree nodes preferentially inter-connect & \citep{Colizza2006}\\
\midrule
Network similarity measures\\
\midrule
Network overlap & & \citep{vanWijk2010}\\
Edge overlap & Percentage of common edges in constant density networks & \citep{Weber2013} \\\botrule
\end{tabular}}{}
\end{table}

\subsection{Node metrics}
For more details on the node metrics see \citep{Rubinov2010}. 
\begin{table}[!t]
\processtable{Formulas for node metrics.\label{Tab:02}}
{\begin{tabular}{ll}
\midrule
Degree & $K_i = \sum_{j=1}^{n}{A_{ij}}$\\
Clustering coefficient & $C_i = 2*e_i / K_i ( K_i -1 )$\\
Path length & $L = 1/N(N-1) \sum_{ij \in n, i \neq j}{d_{ij}}$\\
Global efficiency & $E_{glob} = E(G) = 1/N(N-1) \sum_{i \neq j \in G}{1/d_{ij}}$\\
Local efficiency & $E_{loc} = 1/N \sum_{i \in n}{E(G_i)}$\\\botrule
\end{tabular}}{}
\end{table}

\subsection{Whole-graph metrics}
More formulas go here.

\subsection{Network Similarity}
Use these to examine methods that extract sub-networks such as rich-club and constant density networks.

%\textbf{Figure 1.}{ Enter the caption for your figure here.  Repeat as  necessary for each of your figures.}\label{fig:01}% Don't add the figures in the LaTeX files, please upload them when submitting the article. Frontiers will add the figures at the end of the provisional pdf.

\subsection{Tractograpy}
Algorithms, parameters

\subsection{Matrix Derivation}
Turning streamlines into nice N x N matrices

\subsection{Neuroimaging Data}
The Multi-Modal MRI Reproducibility Resource~\citep{Landman2011}
provides a test-retest data set consisting of 21 subjects with two
time points each for which T1-weighted anatomical images and diffusion
tensor images were acquired. Other image types, not examined here,
were also acquired making this data useful for future examinations of
structure and function.  A population averaged template and manually
defined cortical labels for one time point for each subject are
available as part of the Mindboggle-101 dataset \citep{Klein2012}.

\subsection{Image Analysis}

\textbf{Minimal processing}
\begin{enumerate}
\item Intra-subject registration for label transfer
\item B0-T1 registration
\item DTI reconstruction
\item MTR ?
\end{enumerate}
For each subject, cortical labels were manually defined on the
T1-weighted image from a single time-point. In order to label the
addtional time-point, we use the ANTs Toolkit to find an affine
mapping between the T1-weighted images using the cross-correlation
metric. The cortical labels are then mapped into the unlabeled image
using nearest-neighbor interpolation.



\textbf{Optional processing}
\begin{enumerate}
\item Registration to template ( for additional label sets )
\end{enumerate}

\textbf{Future processing}
\begin{enumerate}
\item Cortical thickness
\item BOLD / ASL / etc
\end{enumerate}


%\end{methods}

\section{Results}
%Results: This section may be divided by subheadings. Footnotes should not be used and have to be transferred into the main text
Overview of what we found


%\begin{table}[!t]
%\processtable{Resolution Requirements for the figures\label{Tab:01}}
%{\begin{tabular}{lllll}\toprule
%Image Type & Description & Format & Color Mode & Resolution\\\midrule
%Line Art & An image composed of lines and text,  & TIFF, EPS, JPEG & RGB, Bitmap & 900 - 1200 dpi\\
%          & which does not contain tonal or shaded areas.& & &\\
%         Halftone & A continuous tone photograph, which contains no text. & TIFF, EPS, JPEG & RGB, Grayscale & 300 dpi\\
%Combination & Image contains halftone + text or line art elements. & TIFF, EPS, JPEG & RGB,Grayscale & 600 - 900 dpi\\\botrule
%\end{tabular}}{This is a footnote}
%\end{table}

%\begin{equation}
%\sum x+ y =Z\label{eq:01}
%\end{equation}

%\textbf{Table\ref{Tab:01}} shows the resolution requirements for the figures. The figures must be legible:
%\begin{enumerate}
%\item The smallest visible text is no less than 8 points in height, when viewed at actual size.
%\item Solid lines are not broken up.
%\item Image areas are not pixelated or stair stepped.
%\item Text is legible and of high quality.
%\item Any lines in the graphic are no smaller than 2 points width.
%\end{enumerate}

%\textbf{Figure 2.}{ Enter the caption for your figure here.  Repeat as  necessary for each of your figures.}\label{fig:02}

\section{Discussion}
% Discussion: This section may be divided by subheadings. Discussions should cover the key findings of the study: discuss any prior art related to the subject so to place the novelty of the discovery in the appropriate context; discuss the potential short-comings and limitations on their interpretations; discuss their integration into the current understanding of the problem and how this advances the current views; speculate on the future direction of the research and freely postulate theories that could be tested in the future.

\subsection{Data Sharing}
%Frontiers supports the policy of data sharing, and authors are advised to make freely available any materials and information described in their article, and any data relevant to the article (while not compromising confidentiality in the context of human-subject research) that may be reasonably requested by others for the purpose of academic and non-commercial research. In regards to deposition of data and data sharing through databases, Frontiers urges authors to comply with the current best practices within their discipline.

\section*{Disclosure/Conflict-of-Interest Statement}
%All relationships financial, commercial or otherwise that might be perceived by the academic community as representing a potential conflict of interest must be described. If no such relationship exists, authors will be asked to declare that the research was conducted in the absence of any commercial or financial relationships that could be construed as a potential conflict of interest.
The authors declare that the research was conducted in the absence of any commercial or financial relationships that could be construed as a potential conflict of interest.

\section*{Acknowledgement}
Shoutouts to our peeps

\paragraph{Funding\textcolon} Shoutout to our peep\$

\section*{Supplemental Data}
Maybe need this, maybe not

\bibliographystyle{frontiersinSCNS} % for Science articles
%\bibliographystyle{frontiersinMED} % for Medicine articles
\bibliography{priorwork}

%\begin{thebibliography}{}

%\bibitem[Bofelli {et~al}., 2000]{Boffelli03} Bofelli,F., Name2, Name3 (2003) Article title, {\it Journal Name}, 199, 133-154.

%\bibitem[Bag {et~al}., 2001]{Bag01} Bag,M., Name2, Name3 (2001) Article title, {\it Journal Name}, 99, 33-54.

%\end{thebibliography}
\end{document}
